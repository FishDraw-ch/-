

\chapter{集合论}
\section{朴素集合论}
本节中,我们使用高中阶段学习的朴素的集合概念. 严谨的集合概念将在\ref{ZF公理体系}节中简介.
\subsection{基本概念}
\begin{definition}[幂集]
    给定一个集合\(A\),其幂集定义为
\[   \pws{A} \defi \z\{   A\,\tx{的所有子集}  \y\}.   \]
\end{definition}

根据定义立得
\begin{equation*}
    \begin{array}{c}
        A\in\pws{A}; \\
        \vd\in\pws{A}; \\
        x\in A \implies \z\{x\y\}\in\pws{A}.
    \end{array}
\end{equation*}
故\(\pws{A}\)肯定不是空集.

\begin{example}
    \[
    \pws{\vd}=\z\{  \vd \y\};\;\pws{\z\{ \vd \y\}}=\z\{ \vd, \z\{ \vd \y\} \y\}.
    \]
\end{example}

\begin{definition}{含于}
    任二集合 $A$、$B$,定义 \[  A\sb B  \iff  \forall\, x\in A ;\, x\in B. \] 特别地,$A\sb B,\,B\sb A \implies A=B.$
\end{definition}
\begin{remark}
    这是集合包含关系的定义,也是我们证明集合包含关系最最常用的方法. 请抛弃你高中所练习的那些奇技淫巧,回归定义.
\end{remark}
\vspace{1cm}

\begin{definition}[集合的基本运算]\label{集合的基本运算定义}
    给定全集\(\Omega\),给定集合\(A\,,\,B\in\pws{\Omega}\),定义以下运算:
\begin{itemize}
    \item \(A\)和\(B\)的并:\[   A\cup B \defi \z\{  x\;;\;x\in A\,\tx{或}\,x\in B \y\}.  \]
    \item \(A\)和\(B\)的交:\[   A\cap B \defi \z\{  x\;;\;x\in A\,\tx{且}\,x\in B \y\}.  \]
    \item \(B\)在\(A\)中的差:\[  A\m B \defi \z\{ x\in A\;;\;x\notin B \y\}.  \]
    \item \(A\)和\(B\)的对称差:\[  A\triangle B \defi \z( A\m B \y) \cup \z( B\m A \y).  \]
    \item \(A\)的补集:\[  A\c \defi \Omega\m A. \]
\end{itemize}
\end{definition}

\begin{remark}
    我们用差集定义了补集. 但其实补集和差集的引入是平等的,没有先后顺序,因为我们也可以用补集来定义差集:\[ A\c \defi \z\{ x\in\Omega\;;\;x\notin A \y\},\, A\m B \defi A \cap B\c.  \]
\end{remark}

\begin{exercise}
    在高中,我们学习了用Venn图\footnote{这是英国数学家John Venn在1880年引入的. 但类似的想法早已出现在德意志剧作家、神学家Christian Weise死后四年——1712年出版的其著作中.}来表示集合,请尝试用Venn图来表示两个集合的对称差.
\end{exercise}
\vspace{1cm}



\begin{theorem}{集合的基本运算律}{A.1}
    若$A$,$B$,$C$是全集$\Omega$上的三个集合,则由定义显然有\footnote{若你觉得不显然,那请你自己证一遍,作为习题.}:
\begin{enumerate}[(i)]
    \item \[    A\sb B \iff A \cup B = B \iff A\cap B=A.   \]
    \item \[    A=\z(A\cap B\y)\cup\z(A\m B\y) = \z(A\cap B\y)\cup\z(A\cap B\c\y).   \] \label{集合运算技巧}
    \item 交换律: \[   A\cup B=B\cup A,\, A\cap B=B\cap A. \] \label{集合交换律}
    \item 结合律: \[   A\cup\z(B\cup C\y)=\z( A\cup B \y)\cup C,\, A\cap\z(B\cap C\y)=\z( A\cap B \y)\cap C.   \] 于是我们可以省略上式中的括号. \label{集合结合律}
    \item 分配律: \[   A\cup\z(B\cap C\y)= \z(A\cup B\y)\cap\z(A\cup C\y),\,A\cap\z(B\cup C\y)= \z(A\cap B\y)\cup\z(A\cap C\y). \] \label{集合分配律}
    \item de Morgan律: \[    \z(A\cup B\y)\c=A\c\cap B\c,\,\z(A\cap B\y)\c=A\c\cup B\c.   \] \label{de Morgan律}
\end{enumerate}
\end{theorem}

\begin{remark}
    在集合的计算中,如果遭遇卡壳,\ref{集合运算技巧}的使用可能会让你柳暗花明又一村. 在正文中我们会体会到这一点.
\end{remark}
\begin{remark}
    由并集和补集的交换律和结合律,以下式子是良定的:
    \[   \bcup_{\alpha\in I} A_\alpha,\;\bcap_{\alpha\in I} A_\alpha;  \]
    于是我们可以把定理\ref{thm:A.1}中的最后两条推广到一般情况,我们将在\ref{集合任意交并的小节}小节介绍.
\end{remark}


\subsection{Cartesian积I}
众所周知,也可以由\ref{ZF公理体系}节中的公理推知,集合中的元素之间是平等、无序的. 我们可以在无序之上构造有序吗?
\begin{definition}[有序对]
    任二数学对象$a$和$b$的有序对记为$\z(a,b\y)$,定义为:\[  \z(a,b\y) \defi \z\{ \z\{a\y\},\z\{a,b\y\} \y\}.    \] 
    这里称$a$为有序对的第一坐标,$b$为有序对的第二坐标. 当$a=b$时,$\z(a,b\y)=\z\{\z\{a\y\}\y\}$.
    
    我们还可以递归地定义$n\in\N^*$个元素$\z\{ x^i \y\}_{i=1}^n$的有序对:\[   \z(  x^1,\cdots,x^n \y) \defi \z( \cdots \z(\z( x^1,x^2\y),x^3\y)\cdots x^n\y),  \] 并称$x^i$是有序对的第$i$坐标.
\end{definition}
\begin{remark}
    注意到我们用“包含”来构造顺序;即某元素比另一元素更“后”,则在某种意义上,靠“后”的元素包含靠“前”的元素. 而这与自然数,乃至序数的构造如出一辙. 具体参见\ref{ZF公理体系}中的“无穷公理”.
\end{remark}
\begin{theorem}
    任二有序对$\z(a_1,b_1\y)$和$\z(a_2,b_2\y)$,那么\[  \z(a_1,b_1\y) =  \z(a_2,b_2\y)  \iff a_1=a_2\,\tx{且}\,b_1=b_2.  \]
\end{theorem}
\begin{proof}
    “$\impliedby$”显然.
    
    “$\implies$”:假设$\z(a_1,b_1\y) =  \z(a_2,b_2\y)$成立. 再若$a_1=b_1$,则
    \[   \z\{ \z\{a_2\y\},\z\{a_2,b_2\y\}\y\} = \z(a_2,b_2\y)=\z(a_1,b_1\y)=\z\{\z\{a_1\y\}\y\}.  \]
    此时必有$\z\{a_2\y\}=\z\{a_2,b_2\y\}=\z\{a_1\y\}$,从而$a_2=b_2=a_1=b_1$.

    现在,再若$a_1\ne b_1$,则
    \[   \z\{ \z\{a_2\y\},\z\{a_2,b_2\y\}\y\} = \z(a_2,b_2\y)=\z(a_1,b_1\y)= \z\{ \z\{a_1\y\},\z\{a_1,b_1\y\}\y\}.  \]
    等号最右边的集合有两个元素,所以等号最左边的集合也必须有两个元素,从而$a_2\ne b_2$. 因此\[\z\{a_2\y\}=\z\{a_1\y\},\,\z\{a_2,b_2\y\}=\z\{a_1,b_1\y\}.\] 此即$a_1=a_2\,\tx{且}\,b_1=b_2$.
\end{proof}


\begin{definition}[Cartesian积]
    $A$和$B$是任二集合. 定义它们的Cartesian积为:\[   A\times B \defi \z\{ \z(a,b\y)\;;\;a\in A\,\tx{且}\,b\in B \y\}.   \] 同理,我们可以递归地定义$n\in\N^*$个集合$\z\{ X_i \y\}_{i=1}^n$的Cartesian积:\[    \prod_{i=1}^n X_i = \z(  X_1\times\cdots\times X_{n-1}  \y)\times X_n = X_1\times\cdots\times X_n.   \] 括号可以省略因为Cartesian积具有结合性. 证明下附. 
\end{definition}
\begin{example}
    若$A=\z\{a,b\y\}$,$B=\z\{i,j,k\y\}$,则\[   A\times B = \z\{    \z(a,i\y),\z(b,i\y),\z(a,j\y),\z(b,j\y),\z(a,k\y),\z(b,k\y)   \y\}.   \]
\end{example}   
\begin{exercise}
    证明Cartesian积具有结合性,即对任三集合$A$、$B$和$C$,有\[\z(A\times B\y)\times C = A\times\z(B\times C\y).\]
\end{exercise}
\begin{solution}
    事实上,只需证明 $A\times \z(B\times C\y)$同构于 $\z(A\times B\y)\times C$即可.
    \begin{align*}
        A\times \z(B\times C\y) &= \z\{   \z(  \z(a,b\y),c  \y)=\z(a,b,c\y) \;;\; a\in A,\, b\in B,\, c\in C  \y\} \\
        &\cong \z\{  \z( a,\z(b,c\y) \y) \;;\; a\in A,\, b\in B,\, c\in C  \y\}  = \z(A\times B\y)\times C.
    \end{align*}
    证毕.
\end{solution}
\begin{exercise}
    证明若$A\ne B$,则$A\times B\ne B\times A$和\[  A\times B=\vd \iff A=\vd\,\tx{或}\,B=\vd.  \]
\end{exercise}

\subsection{映射}
我们由Cartesian积的概念来严谨地定义映射.
\begin{definition}[对应法则]\label{对应法则定义}
    设$C$和$D$是任两个集合. 一个对应法则是指一个集合$L\sb C\times D$,满足
    \[   \z(c,d\y)\in L,\,\z(c,d'\y)\in L \implies d=d'.   \]
    定义对应法则$L$的定义域和值域为
    \begin{align*}
        \mathbf{Dom}\z(L\y) \equiv L\,\tx{的定义域}\, &\defi \z\{   c\in C\;;\; \exists\, d\in D,\, \z(  c,d \y)\in L    \y\}; \\
        \mathbf{Im}\z(L\y) \equiv L\,\tx{的值域}\, &\defi \z\{   d\in D\;;\; \exists\, c\in C,\, \z(  c,d \y)\in L    \y\}.
    \end{align*}
\end{definition}
\begin{definition}[映射]
    一个映射$f$被定义为一个有序对:
    \[f\defi \z(   L,B  \y),\]
    其中$B\sp \mathbf{Im}\z( L \y)$,称为“$f$的陪域”;且我们定义
    \begin{align*}
        \mathbf{Dom}\z(f\y) \equiv f\,\tx{的定义域}\, &\defi \mathbf{Dom}\z(L\y); \\
        \mathbf{Im}\z(f\y) \equiv f\z( \mathbf{Dom}\z(f\y) \y) \equiv f\,\tx{的值域}\, &\defi \mathbf{Im}\z(L\y).
    \end{align*}
    我们引入记号\footnote{这组记号由Euler在1734年引入.}:
    \begin{align*}
        f\, :\quad \; A &\ra B \\
        a &\mapsto f\z(a\y),
    \end{align*}
    其中$A\equiv\mathbf{Dom}\z(f\y)$是$f$的定义域,$B$是$f$的陪域,$f\z(a\y)$是$B$中满足$\z( a,f\z(a\y) \y)\in L$的那个唯一元素.

    特别地,如果映射 $f$的陪域 $B$是一个数集,则称映射 $f$为一个函数.
\end{definition}
\begin{remark}
    我们为什么要在值域的基础上额外引入一个陪域呢?这是因为“值域”可以理解为函数值的“精确”范围,而陪域可以理解为函数值的“粗糙”范围;而很多映射的值域没法轻松求出来,而陪域肯定都很好求,因为大不了我们把陪域取得很广很广,总能把值域盖住. 
\end{remark}
\begin{example}
    Lambert的$W$函数定义为:
    \[   W\;:\; \z[-\frac1\ee,+\wq\y) \ra \R\;;\,\tx{满足}\;W\z(x\y)\ee^{W\z(x\y)} = x,\,W\z(x\y)\,\tx{单增}.   \]
    其陪域显然为$\R$,试求其值域,可以得到$\z[ -1,+\wq \y)$,请你自己求一下来感受难度.
\end{example}
\vspace{1cm}

\begin{proposition}[关于映射]
    我们罗列一下映射的相关内容.
    \begin{itemize}
        \item 任一映射$F\;:\; A\ra B$,任一$A_0\sb A$,定义$f$在$A-0$上的限制为映射:\[  \z. f \y|_{A_0} \;:\; A_0 \ra B.    \]
        \item 映射最重要的性质恐怕就是单射、满射和双射了. 我们有:
        \begin{itemize}
            \item $f$是单设当且仅当 \[   \forall\, a,b\in A;\; f\z(a\y)=f\z(b\y) \implies a=b.  \]
            \item $f$是满射当且仅当 \[   \forall\, b\in B,\,\exists\, a\in A,\,f\z(a\y)=b.  \]
            \item $f$是双射当且仅当它既是单的又是满的.
        \end{itemize}
        \item 定义$f$和$g$的复合如下:
        \[   g\circ f \;:\; A\ra C,\,a\mapsto c;\;\exists\,b\in B,\;f\z(a\y)=b,\,g\z(b\y)=c.   \] 
        显然$g\circ f$仅当$\mathbf{Im}\z(f\y)\sb\mathbf{Dom}\z(g\y)$时有定义. 注意到一般情况下$f\circ g \ne g\circ f$.
        \item 若$f$是双射,则可定义其逆映射$f\inv\;:\;B\ra A$如下:
        \[   f\inv\z(b\y)=a \iff f\z(a\y)=b.   \]
    \end{itemize}
\end{proposition}
我们紧接着给出一些常用的基本映射.
\begin{definition}[恒等映射和常值映射]
    我们有两个特别重要的映射:
         \begin{itemize}
             \item 恒等映射:$\id_A \;:\; A\ra A\,,\,a\mapsto a.$
             \item 常值映射:$c \;:\; A\ra B\,,\,a\mapsto b$,其中$b$是固定的.
         \end{itemize}
\end{definition}
\begin{definition}[Gau\ss 取整函数]
    对于任意 $x\in\R$,定义函数
    \begin{align*}
        \z\lfloor \,\cdot\, \y\rfloor \;:\quad \R &\ra \Z \\
        x &\mapsto \z\lfloor x \y\rfloor \,\tx{为不超过}\,x\,\tx{的最大整数.}
    \end{align*}
    即 $x$的整数部分. 同理可定义
    \begin{align*}
        \z\lceil \,\cdot\, \y\rceil \;:\quad \R &\ra \Z \\
        x &\mapsto \z\lceil x \y\rceil \,\tx{为不小于}\,x\,\tx{的最大整数.}
    \end{align*}
    也可定义
    \begin{align*}
        \z\{ \,\cdot\, \y\} \;:\quad \R &\ra \Z \\
        x &\mapsto \z\{ x \y\}=x-\z\lfloor x \y\rfloor \,\tx{,即}\,x\,\tx{的小数部分.}
    \end{align*}
    注意,请不要把“取小数”函数的花括号与集合的花括号混淆.
\end{definition}
\begin{definition}[示性函数]\label{示性函数}
    给定任二集合$A_0$和$A$,要求$A_0\sb A$,则可定义$A_0$在$A$上的示性函数$\chi_{A_0}\defi$
    \begin{align*}
        \chi_{A_0} \;:\qquad A &\ra \z\{0,1\y\} \\
        A\ni a &\mapsto \chi_{A_0}\z(a\y) = \begin{cases}
            1, \quad &a\in A_0;\\
            0, \quad &a\in A\m A_0
        \end{cases}.
    \end{align*}
    示性函数在测度论中使用极其频繁.

    由定义显然示性函数与全集上的子集是一一对应的.
\end{definition}

\begin{theorem}[单射和满射的一般定义]{单射和满射的一般定义}
    \begin{enumerate}
        \item 任一映射$f$,若$\exists\, g \;:\; B\ra A;\,g\circ f = \id_A$,即 \[   \forall\,a\in A;\;g\z(f\z(a\y)\y)=a;   \] 那么$f$是单射.
        \item 任一映射$f$,若$\exists\, h \;:\; B\ra A;\,f\circ h = \id_B$,即 \[   \forall\,a\in A;\;f\z(h\z(b\y)\y)=b;   \] 那么$f$是满射.
    \end{enumerate}
\end{theorem}
\begin{proof}
    \begin{enumerate}
        \item \(    f\z(a\y)=f\z(b\y) \implies a=g\z( f\z(a \y)  \y)=g\z( f\z(b \y) \y) = b.   \)
        \item \(    \forall\, b\in B,\,\exists\, a=h\z(b\y)\in A,\,f\z(a\y)=f\z(h\z(b\y)\y)=b.    \)
    \end{enumerate}
\end{proof}
于是我们也可以用定理\ref{thm:单射和满射的一般定义}中的式子来定义单射和满射. 这样的定义方式会在范畴论中有用.

\begin{definition}[像集和原像集]
    给定映射$f\;:\;A\ra B$,给定$A_0\sb A,\,B_0\sb B$,定义:
    \begin{enumerate}
        \item $A_0$在$f$下的像集$\equiv f\z(A_0\y) \defi \z\{ f\z(a\y)\;;\;a\in A_0  \y\}\sb B$.
        \item $B_0$在$f$下的原像集$\equiv f\inv\z(B_0\y) \defi \z\{ a\;;\;f\z(a\y)\in B_0  \y\}\sb A$. 
        特别地,如果$B_0=\z\{b\y\}$,则符号$f\inv\z(b\y)\defi f\inv\z(  \z\{b\y\}  \y)$在一般情况下不在是一个单元素集;如果$f$本身是一个双射,则符号$f\inv\z(b\y)$与之前的定义相同. 这里虽然出现了符号的滥用,但是这个滥用在实践中由语境可以轻松判断.
    \end{enumerate}
    显然有以下包含关系式成立:\[     A_0\sb f\inv\z(  f\z( A_0 \y)  \y),\,B_0\sp f\z(  f\inv\z(   B_0  \y)  \y).   \]
\end{definition}
我们可以找到例子说明上述二包含关系的等号不一定成立.
\begin{example}
    令
    \begin{align*}
        f\;:\; \R &\ra \R, \\
        x &\mapsto 3x^2+2.
    \end{align*}
    取$A_0=\z[0,1\y],\,B_0=\z[0,5\y]$,则
    \[  f\inv\z(  f\z( A_0 \y)  \y)=\z[-1,1\y]\supset A_0 ,\, f\z(  f\inv\z(   B_0  \y)  \y)=\z[2,5\y]\subset B_0.  \]
\end{example}

\begin{proposition}
    我们不加证明地给出以下关系式(如果你是初学者,我强烈建议你全部自己证一遍):

    若$f\;:\;A\ra B$是映射,且$A_0,\,A_1\sb A\,;\;B_0,\,B_1\sb B$,有
    \begin{enumerate}[(i)]
        \item $B_0\sb B_1 \implies f\inv\z(B_0\y) \sb f\inv\z(B_1\y)$;
        \item $f\inv \z(  B_0\cup B_1  \y)= f\inv\z(B_0\y)\cup f\inv\z(B_1\y)$;
        \item $f\inv \z(  B_0\cap B_1  \y)= f\inv\z(B_0\y)\cap f\inv\z(B_1\y)$;
        \item $f\inv \z(  B_0\m B_1  \y)= f\inv\z(B_0\y)\m f\inv\z(B_1\y)$;
        \item $A_0\sb A_1 \implies f\z(A_0\y)\sb f\z(A_1\y)$;
        \item $f\z(A_0\cup A_1\y)=f\z(A_0\y)\cup f\z(A_1\y)$;
        \item $f\z(A_0\cap A_1\y)\sb f\z(A_0\y)\cap f\z(A_1\y)$;
        \item $f\z(A_0\m A_1\y)\sp f\z(A_0\y)\m f\z(A_1\y)$.
    \end{enumerate}
\end{proposition}
\begin{example}
    我们给出(vii)和(viii)等号不成立的例子. 取常值映射 $\forall\, a\in A,\,f\z(a\y)=b\in B$为给定常元素. 不妨设 $A$至少含有两个不同元素$a_0$和$a_1$.
    \begin{enumerate}
        \item 取 $A_0=\z\{a_0\y\},\,A_1=\z\{a_1\y\}$,则 $A_0\cap A_1=\vd$. 于是
        \begin{align*}
            f\z(A_0\cap A_1\y) &= f\z(\vd\y)=\vd \\
            &\subset \z\{b\y\}=\z\{b\y\}\cap\z\{b\y\} \\
            &= f\z(A_0\y) \cap f\z(A_1\y).
        \end{align*}
        \item 取 $A_0=\z\{a_0,a_1\y\},\,A_1=\z\{a_1\y\}$. 则 $A_0\m A_1=\z\{a_0\y\},\,f\z(A_0\y)=f\z(A_1\y)=\z\{b\y\}$. 于是
        \[   f\z(A_0\m A_1\y)=\z\{b\y\}\supset\vd=f\z(A_0\y)\m f\z(A_1\y).  \]
    \end{enumerate}
\end{example}
\begin{remark}
    这种不对称性来源于对应法则的定义\ref{对应法则定义}中用到了不对称的有序对,且条件\[\z(c,d\y)\in L,\,\z(c,d'\y)\in L \implies d=d'\]只对有序对第二坐标位置上的元素有限制.
\end{remark}

\vspace{1cm}
\begin{definition}\label{映射集的定义}
    任二集合$A$、$B$,我们将全体$A\ra B$的映射所成的集合记为$B^A$.
\end{definition}
\begin{example}
    我们再来考察一下一个集合$A$的幂集 $\pws{A}$. 我们将幂集记为 $2$的幂次的形式不仅仅是因为在 $A$有限的情况下,$A$的幂集的元素个数正好等于 $2$的 $A$的元素个数次幂,更是因为由以上定义,若把底数“$2$”看成一个二元集 $\z\{ 0,1 \y\}$,则$\z\{ 0,1 \y\}^A$就是全体 $A$的子集的示性函数之集合. 而示性函数与其子集一一对应,因此 $\z\{ 0,1 \y\}^A$与 $\pws{A}$同构.
\end{example}
\begin{example}
    任一实数列$\z\{ a_n \y\}_{n\in\N}$都可以看作一个从$\N$到$\R$的映射,因此全体实数列构成的集合就记作$\R^\N$. 同理,全体复数列构成的集合就是$\C^\N$.
\end{example}

\vspace{1cm}
\begin{remark}
    在测度论中,我们会经常用到形如$f\inv\z(-\wq,c\y]$、$f\inv\z[c,+\wq\y)$或$f\inv\z(c,+\wq\y)$之类的集合.
\end{remark}



\subsection{集合的任意交、并和Cartesian积II}\label{集合任意交并的小节}
有了映射的概念,我们可以来定义数学中非常重要的一个概念:指标.

我们将以集合为元素的集合称为集族.
\begin{definition}[指标映射]
    \begin{itemize}
        \item 设非空集族$\msA$. 定义$\msA$的指标映射是满射:
        \[  f\;:\;I\ra\msA.  \] 其中$I$是一个任意集合,被称为指标集.
        \item 定义有序对$\z( \msA,f \y)$为“$\msA$的指标类”.
        \item 给定$\alpha\in I$,把集合$f\z(\alpha\y)\in\msA$记作$A_\alpha$,并把$\msA$的指标类记作$\z\{ A_\alpha\y \}_{\alpha\in I}$.
    \end{itemize}
\end{definition}
有了指标的概念,我们可以推广定理\ref{thm:A.1}中的\ref{集合分配律}和\ref{de Morgan律}.
\begin{definition}[集合的任意交、并]
    由定理\ref{thm:A.1}中的\ref{集合交换律}交换律和\ref{集合结合律}结合律,以下定义是良定的:
    \begin{align*}
        \bcup_{\alpha\in I} A_\alpha &\defi \z\{     x\;;\;\exists\,\alpha\in I,\,x\in A_\alpha.   \y\}, \\
        \bcap_{\alpha\in I} A_\alpha &\defi \z\{     x\;;\;\forall\,\alpha\in I,\,x\in A_\alpha.   \y\}.
    \end{align*}
    特别地,若
    \begin{itemize}
        \item $I=\z\{ 1,\cdots,n  \y\}$,则记
        \[  \bcup_{\alpha\in I} A_\alpha=\bcup_{i=1}^n A_i\,,\; \bcap_{\alpha\in I} A_\alpha=\bcap_{i=1}^n A_i;  \]
        \item $I=\N$,则记
        \[  \bcup_{\alpha\in I} A_\alpha=\bcup_{i\in\N} A_i\,,\; \bcap_{\alpha\in I} A_\alpha=\bcap_{i\in\N} A_i.  \]
    \end{itemize}
\end{definition}
下面推广定理\ref{thm:A.1}中的\ref{集合分配律}和\ref{de Morgan律}.
\begin{theorem}[一般的交、并的分配律和de Morgan律]\label{分配律和de Morgan律推广}
    \begin{itemize}
        \item 分配律:\begin{align*}
            \bcup_{j=1}^n \z(  \bcap_{k_j=1}^{m_j}  E_j^{k_j}  \y) &= \bcap_{k_1=1}^{m_1} \bcap_{k_2=1}^{m_2} \cdots \bcap_{k_n=1}^{m_n} \z( \bcup_{j=1}^n  E_j^{k_j} \y)\;,\\
            \bcap_{j=1}^n \z(  \bcup_{k_j=1}^{m_j}  E_j^{k_j}  \y) &= \bcup_{k_1=1}^{m_1} \bcup_{k_2=1}^{m_2} \cdots \bcup_{k_n=1}^{m_n} \z( \bcap_{j=1}^n  E_j^{k_j} \y)\;.
        \end{align*}
        \item de Morgan律:\[ \z(  \bcup_{\alpha\in I} A_\alpha  \y)\c=\bcap_{\alpha\in I} A_\alpha\c \;,\;\; \z(  \bcap_{\alpha\in I} A_\alpha  \y)\c=\bcup_{\alpha\in I} A_\alpha\c \,.\]
    \end{itemize}
\end{theorem}
\vspace{1cm}

下面我们引入数组的概念来推广Cartesian积.
\begin{definition}[数组]
设$m\in\N^*$,给定集合$X$,
\begin{itemize}
    \item 定义$X$的$m$-数组为映射
    \[  \hat{x} \;:\; \z\{  1,\cdots,m \y\} \ra X ,   \] 对每个$i\in\z\{  1,\cdots,m \y\}$,记$x_i=\hat{x}\z(i\y)$,并称为$\hat{x}$的$i$坐标. 这样,我们就可以把一个$m$-数组映射表示成$\hat{x}=\z(  x_1,\cdots,x_n \y)$.
    \item 定义$X$的$\omega$-数组为映射
    \[  \hat{x} \;:\; \N^*\ra X\,,\; n\mapsto x_n=\hat{x}\z(n\y),  \]
    这样,我们就可以把一个$\omega$-数组映射表示成$\hat{x}=\z( x_n \y)_{n\in\N^*}$.
    \item 一般地,定义$X$的$I$-数组为映射
    \[  \hat{x} \;:\; I\ra X\,,\; \alpha\mapsto x_\alpha=\hat{x}\z(\alpha\y),  \]
    这样,我们就可以把一个$I$-数组映射表示成$\hat{x}=\z( x_\alpha \y)_{\alpha\in I}$.
\end{itemize}
\end{definition}
\begin{definition}[一般的Cartesian积]
    设$\z\{ A_\alpha \y\}_{\alpha\in I}$是集族的指标类,$m\in\N^*$;
    \begin{itemize}
        \item 若$I=\z\{  1,\cdots,m \y\}$,令$X=\displaystyle \bcup_{i=1}^m A_i $,定义这个指标类的Cartesian积,写作\[  \prod_{i=1}^m  A_i\,,  \] 定义为$X$的全体满足$x_i\in A_i$的$m$-数组构成的集合.
        \item 若$I=\N^*$,令$X=\displaystyle \bcup_{n\in\N^*} A_n $,定义这个指标类的Cartesian积,写作\[  \prod_{n\in\N^*}^m  A_n\,,  \] 定义为$X$的全体满足$x_n\in A_n$的$\omega$-数组构成的集合.
        \item 若$I$是一般集合,令$X=\displaystyle \bcup_{\alpha\in I} A_\alpha $,定义这个指标类的Cartesian积,写作\[  \prod_{\alpha\in I}  A_\alpha\,,  \] 定义为$X$的全体满足$x_\alpha\in A_\alpha$的$I$-数组构成的集合.
    \end{itemize}
\end{definition}
\begin{remark}
    结合定义\ref{映射集的定义},显然有 \[   \prod_{\alpha\in I}  A_\alpha \sb \z( \bcup_{\alpha\in I} A_\alpha  \y)^I,  \] 其中等号取到当且仅当每个$A_\alpha$都相等.
\end{remark}




\section{Zermelo-Fraenkel公理体系}\label{ZF公理体系}


\begin{axiom}[无穷公理]
    
\end{axiom}
\begin{axiom}[幂集公理]
    
\end{axiom}
\begin{axiom}[外延公理]
    
\end{axiom}
\begin{axiom}[替代公理模式]
    
\end{axiom}
\begin{axiom}[分离公理模式]
    
\end{axiom}
\begin{axiom}[正则公理]
    
\end{axiom}
\begin{axiom}[并集公理]
    
\end{axiom}
\begin{axiom}[配对公理]
    
\end{axiom}


\subsection{选择公理vs.决定公理}\label{选择公理和决定公理}

\begin{axiom}[选择公理]
    
\end{axiom}
\begin{axiom}[决定公理]
    
\end{axiom}

\section{基数vs.序数}

\subsection{基数}\label{基数}



\begin{postulate}[连续统假设]
    
\end{postulate}


\subsection{序数}
\begin{definition}[序数]
    
\end{definition}







