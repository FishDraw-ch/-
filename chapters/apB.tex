\chapter{代数学}

\section{二元关系}
本小节将小学学过的“$=$”“<”等关系从定义在数集上推广到定义在一般集合上.
\subsection{等价关系}

\subsection{序关系}
\subsubsection{偏序关系}
\subsubsection{全序关系}
\subsubsection{良序关系}
\begin{axiom}[良序原理]
    
\end{axiom}
\begin{axiom}[Zorn引理]
    
\end{axiom}
\begin{theorem}
    选择公理、良序原理、Zorn引理相互等价.
\end{theorem}
\begin{theorem}
    选择公理独立于Zermelo-Fraenkel公理体系. 即在Zermelo-Fraenkel公理体系下,选择公理无法被证明也无法被证伪.
\end{theorem}
\begin{corollary}
    选择公理、良序原理、Zorn引理均独立于Zermelo-Fraenkel公理体系.
\end{corollary}


\section{集合上的代数结构}

\subsection{群}
\subsubsection{子集的代数运算}
\begin{definition}[子集和元素间的代数运算]\label{子集的代数运算}
    
\end{definition}
\begin{theorem}[陪集划分是一种等价划分]\label{陪集划分定理}
    
\end{theorem}
\subsection{环}
\subsection{体}
\subsection{域}
\subsection{模}
\subsection{线性空间}
\subsection{格}





























